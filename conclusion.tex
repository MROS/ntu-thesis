\chapter{總結}

本論文描述了淺狀態區塊鏈的設計,
並探討了多種快取策略的優缺點。特別對於 LRU 快取策略,
我們設計了持久化雜湊加持久化紅黑樹的組合、持久化雜湊與值無關順序樹的組合、
持久化雜湊與持久化堆積的組合,這三種時空間複雜度相當的複合資料結構。

最後進行實驗,
瞭解到在實驗環境中採用這幾種資料結構來實作 LRU 快取策略,
獲取快取時的速度皆顯著高過真實以太坊節點(parity)中直接驗證梅克爾證明的速度,
佐證了淺狀態區塊鏈有機會帶來效能提升。