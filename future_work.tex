\chapter{未來工作}
\label{c:future_work}

\section{快取拆分}

若要讓淺狀態區塊鏈支援智慧合約,單以一個賬戶為快取單位會導致一些問題,
因為每個智慧合約的狀態所佔用的空間可能相差巨大,如果快取中的每一個狀態都非常大,
那整個快取就會佔用很多空間,反之若快取中的狀態都很小,整個快取佔用空間就很小,
當快取佔用的空間不穩定時,節點的維護者就很難為自己的機器設定參數。

我們可以借鑑作業系統的分頁設計,將智慧合約的狀態切割成等體積的多個小塊,
每次快取只會存取到用到的小塊,如此就能夠使快取佔用空間變得穩定。

% \section{隨機化順序樹}
% TODO