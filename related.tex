\chapter{相關研究}
\label{c:related}

\section{Utreexo}
Utreexo~\cite{dryja2019utreexo} 應用梅克爾樹構成的森林設計了一種新的 accumulator ,
基於此種 accumulator ,Utreexo 節點不需要真正的儲存比特幣所有的 UTXO ,
就能夠驗證區塊。

然而,比特幣網路已經存在多年,大幅修改協議使得所有節點改為無狀態並不現實,
Utreexo 提議可以採用一個橋接節點作為比特幣全節點與 Utreexo 節點的中介,
當區塊廣播由 Utreexo 節點到全節點時,會剝離掉證明,反之則補上證明。

該篇論文也提及到,分析比特幣的歷史記錄,
約有 40\% 的 UTXO 會在 20 個區塊內被消耗,
將近 80\% 的 UTXO 會在 1000 個區塊內被消耗,
因此使用少量的記憶體空間來快取最近出現的 UTXO ,就能夠省略傳輸許多證明。

與本研究的差別在於,本研究考慮基於帳號的虛擬貨幣,而非基於 UTXO 的,
同時,本研究也額外考量了分叉時的情況。

\section{Making Data Structure Persistent}
該篇論文~\cite{driscoll1986making}提出了多種算法能夠使得任何基於節點的資料結構半/全持久化(partial/fully persistent),
根據該論文的 node-splitting 算法,
甚至能夠在 $O(1)$ 時間複雜度內完成雙向鏈表的插入、修改。

然而,該論文並沒有提出刪除持久化資料結構的老舊版本的方法,
這使得這些算法難以應用到淺狀態區塊鏈的快取上,
因為無法刪除過去版本將導致快取所需的空間始終無法釋放,
最終耗盡記憶體容量。