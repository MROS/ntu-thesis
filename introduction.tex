\chapter{Introduction}
\label{c:intro}

2008 年,中本聰發佈了比特幣的白皮書~\cite{nakamoto2019bitcoin},
這是一個以區塊鏈為根本的新型虛擬貨幣系統,
這個系統中的每個節點都會保留一個完整的賬本,
如此確實使得該系統的去中心化程度達到相當高的程度,
但不可磨滅的賬本資訊也導致每個節點所需要的儲存空間逐年增加。

在十多年後的現在,人們需要上百 GB 的硬碟空間,
才能夠運行最為知名的虛擬貨幣系統比特幣、以太坊~\cite{wood2014ethereum}的一個節點,
這使得部分由私人維護的節點逐漸不堪負荷。
若退而求其次,僅儲存驗證新區塊所需的資訊,
亦即比特幣中的 UTXO 、以太坊中的世界狀態,
也需要數 GB 到數十 GB 的硬碟空間,
不止是在空間上會對對硬碟的有所要求,速度上也是,
傳統硬碟的隨機存取太慢,驗證以太坊區塊的速度已經跟不上共識生成區塊的速度,
如此迫使以太坊節點的維護者必須購買較為昂貴的固態硬碟,
進一步提高了節點的運行成本。

為此,虛擬貨幣的社群開始嘗試提出一些解決方案,無狀態區塊鏈就是其一。
無狀態區塊鏈僅要求每個節點儲存區塊頭,但每一筆交易必須附帶合法性證明,
如此大幅降低了空間需求,並且無需在硬碟查找過去狀態就能夠驗證交易合法性。

本研究嘗試進一步優化無狀態區塊鏈,
我們讓無狀態區塊鏈的節點有共識的維護快取,
若一筆交易的付款人曾在最近的其他交易中出現過,
就有機會從快取中推知此交易仍然是合法的,
此時便可以在交易中省略合法性證明,進而降低了交易的大小,
也減少廣播時所需要的網路流量。

但是,如何高效的維護快取成了新的問題。每一筆區塊接上區塊鏈時,
都會有一個對應的快取,而臨近區塊上的快取很可能有所重疊,
我們分別為不同的快取策略,最近 k 塊、 LRU ,
設計了持久化的資料結構,並且進行實驗考察了它們的性能。

論文的其餘章節依序如下:第二章講述相關研究,
第三章說明背景以及基礎知識,第四章開始說明淺狀態區塊鏈的設計,
包含快取的策略以及快取使用的資料結構,第五章為實驗的設計與結果,
第六章為未來展望,第七章為總結。