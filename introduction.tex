\chapter{Introduction}
\label{c:intro}

2008 年,中本聰發佈了比特幣的白皮書~\cite{nakamoto2019bitcoin},
這是一個以區塊鏈為根本的新型加密貨幣系統,
這個系統中的每個節點都會保留一個完整的賬本,
如此確實使得該系統的去中心化程度達到相當高的程度,
但不可磨滅的賬本資訊也導致每個節點所需要的儲存空間逐年增加。

在十多年後的現在,人們需要上百 GB 的資料儲存空間,
才能夠運行最為知名的加密貨幣系統比特幣、以太坊~\cite{wood2014ethereum}的一個節點,
這使得部分由私人維護的節點逐漸不堪負荷。
若退而求其次,僅儲存驗證新區塊所需的資訊,
亦即比特幣中的 UTXO 、以太坊中的世界狀態,
也需要數 GB 到數十 GB 的儲存空間,
不止是在空間上對資料儲存裝置有所要求,速度上也是,
傳統硬碟的隨機存取太慢,驗證以太坊區塊的速度已經跟不上共識生成區塊的速度,
如此迫使以太坊節點的維護者必須購買較為昂貴的固態硬碟,
進一步提高了節點的運行成本。

為此,加密貨幣的社群開始嘗試提出一些解決方案,無狀態區塊鏈(stateless blockchain)就是其一。
無狀態區塊鏈僅要求每個節點儲存區塊標頭,但每一筆交易必須附帶合法性證明,
如此大幅降低了空間需求,並且無需在資料儲存裝置中查找過去狀態就能夠驗證交易合法性。

本研究嘗試進一步優化無狀態區塊鏈,
我們讓無狀態區塊鏈的節點有共識的維護快取,
若一筆交易的付款人曾在最近的其他交易中出現過,
就有機會從快取中推知此交易仍然是合法的,
此時便可以在交易中省略合法性證明,進而降低了交易的大小,
也減少廣播時所需要的網路流量。

但是,如何高效的維護快取成了新的問題。每一筆區塊接上區塊鏈時,
都會有一個對應的快取,而臨近區塊上的快取很可能有所重疊,
我們分別為不同的快取策略,最近 k 塊、 LRU ,
設計了持久化的資料結構,並且進行實驗考察了它們的性能。

論文的其餘章節依序如下:
第二章說明背景以及相關工作,第三章開始說明淺狀態區塊鏈(shallow-state blockchain)
\footnote{一般的區塊鏈節點儲存所有歷史狀態;無狀態區塊鏈儲存區塊標頭。
本論文所設計的淺狀態區塊鏈則選擇性地儲存最近存取過的賬戶狀態,
若說一般區塊鏈儲存的是整個大海,那淺狀態區塊鏈儲存的就是淺水區的資訊,
並且隨潮汐不斷更迭。英文也就順勢翻譯成 shallow-state blockchain 。}的設計,
包含快取的策略以及快取使用的資料結構,第四章為實驗的設計與結果,
第五章為未來工作,第六章為總結。