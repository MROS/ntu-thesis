\begin{abstractzh}
基於區塊鏈的加密貨幣網路中,
每個節點都維護一個完整的賬本,
經年累月,賬本的儲存容量越來越高,
維護一個節點的成本也會越來越高。
無狀態區塊鏈是一種針對以上問題的改進方案,
它的每一筆交易中附上了合法性證明,
使得節點不需要進入資料儲存裝置查找過往賬本資訊,
只要驗證交易內附的證明即可,
從而使節點所需儲存的資訊量大幅減少,
然而,要付出額外的網路流量與驗證時間。

在本論文中,我們將在基於賬號的區塊鏈上討論以下兩點:
第一,
一筆交易的合法性很可能由最近出現過的其他交易的合法性推知,
只要快取賬號的訊息,
就可以省略部分交易的證明,
以此來減少網路流量與驗證時間。
第二,
快取替換有多種策略,針對 LRU 策略,
設計了基於不可變紅黑樹與不可變值無關順序樹的兩種資料結構,
並且進行實驗比較了快取命中與快取失效的代價,
模擬現今以太坊的區塊性質,
當快取數十個區塊大小的賬戶資訊量時,
快取命中所需的查找時間仍數倍低於梅克爾 Patricia 證明驗證的時間。


\bigbreak
\noindent
關鍵字:區塊鏈、無狀態區塊鏈、快取、資料結構
\end{abstractzh}

\begin{abstracten}

Cryptocurrency based on blockchain is a peer to peer trading
system. In this decentralized system,
each node maintains a complete ledger which
records entire transaction history of all accounts.
Years after years, requirement to single node will
get higher and higher. Dealing with the issue, stateless
blockchain is a promising approach. To be more specific,
in this system, each transaction is attached with proof
of validity. With this approach, there is no need for a node
to access the data storage device to search past information
of the ledger. Instead, node only verifies the proof attached
to each transaction. Therefore, it will significantly
reduce the space of data storage for each node.
But it will cause additional network traffic and
verification time.

In this paper, we will discuss the following two points
on account-based blockchain.
First,
it is highly possible to infer the validity of
a transaction from another valid transaction appearing recently.
If we cache account information, we can eliminate some
proof in transactions and then reduce network traffic
and verification time.
Second,
there are so many cache replacement polcies.
For LRU polices, we designed two composite data
structures which based on red-black tree and value-independent order tree.
And conduct some experiments to compare the cost of a
cache hit with the penalty of a cache miss.
Simulating today's Ethereum's block property,
when cache size can contain several tens of blocks' information,
the cache searching time is significantly lower than
the time to verify Merkle Patricia proofs.

\bigbreak
\noindent
Keywords: blockchain, stateless blockchain, cache, data structure
\end{abstracten}

% \begin{comment}
% \category{I2.10}{Computing Methodologies}{Artificial Intelligence --
% Vision and Scene Understanding} \category{H5.3}{Information
% Systems}{Information Interfaces and Presentation (HCI) -- Web-based
% Interaction.}

% \terms{Design, Human factors, Performance.}

% \keywords{Region of interest, Visual attention model, Web-based
% games, Benchmarks.}
% \end{comment}
