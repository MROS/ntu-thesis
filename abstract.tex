\begin{abstractzh}
基於區塊鏈的虛擬貨幣是一種點對點的交易系統,
在此去中心化的系統中,
每個節點都維護一個完整的賬本,
其中記錄了所有賬戶的全部交易資料。
這樣的系統具有不容易被竄改資料的特點,
然而此種特性會使得所需要儲存的資料只增不減,
經年累月下來,對單個節點的儲存容量要求只會越來越高,
一旦個體維護的節點逐漸不勘負荷,
將導致此區塊鏈的中心化程度逐步上升。

無狀態區塊鏈是一種針對以上問題的改進方案,
它的每一筆交易中附上了合法性證明,
使得節點不需要進入硬碟查找過往賬本資訊,
只要驗證交易內附的證明即可,
從而使節點所需儲存資訊大幅減少。

在本論文中,我們將討論以下兩點:
第一,並無必要在每一筆交易都附上證明,
只要快取部分訊息,
一筆交易的合法性很可能由最近出現過的其他交易的合法性推知。
第二,當區塊鏈發生分叉時,必須回到分叉點再產生新區塊的快取,
為此我們設計了不可變的資料結構,
使得快取能夠在各種情況下被快速計算。
最後,進行實驗來瞭解數種快取策略與資料結構在不同工作量(workload)下的表現。


\bigbreak
\noindent \textbf{關鍵字:}{\, \makeatletter \@keywordszh \makeatother}
\end{abstractzh}

\begin{abstracten}
This thesis presents a benchmark for region of interest (ROI)
detection. ROI detection has many useful applications and many
algorithms have been proposed to automatically detect ROIs.
Unfortunately, due to the lack of benchmarks, these methods were
often tested on small data sets that are not available to others,
making fair comparisons of these methods difficult. Examples from
many fields have shown that repeatable experiments using published
benchmarks are crucial to the fast advancement of the fields. To
fill the gap, this thesis presents our design for a collaborative
game, called Photoshoot, to collect human ROI annotations for
constructing an ROI benchmark. With this game, we have gathered a
large number of annotations and fused them into aggregated ROI
models. We use these models to evaluate five ROI detection
algorithms quantitatively. Furthermore, by using the benchmark as
training data, learning-based ROI detection algorithms become
viable.

\bigbreak
\noindent \textbf{Keywords:}{\, \makeatletter \@keywordsen \makeatother}
\end{abstracten}

\begin{comment}
\category{I2.10}{Computing Methodologies}{Artificial Intelligence --
Vision and Scene Understanding} \category{H5.3}{Information
Systems}{Information Interfaces and Presentation (HCI) -- Web-based
Interaction.}

\terms{Design, Human factors, Performance.}

\keywords{Region of interest, Visual attention model, Web-based
games, Benchmarks.}
\end{comment}
