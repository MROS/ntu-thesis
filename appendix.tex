\chapter{附錄}

\section{以雜湊 + 最小堆積實作持久化 LRU}

我們也可以巧妙地用雜湊搭配最小堆積(min heap)來實作持久化 LRU 。

在這種構造中,雜湊表的鍵為賬戶地址,值為 ( 時間戳, 賬戶資訊 ) 。
堆積用單向指標來實作,堆積中的每一個節點會儲存賬戶地址以及時間戳。

以下是堆積中一個節點的結構,
在我們所維護的最小堆積中,
父節點的時間戳都小於自己的時間戳。

\begin{lstlisting}
struct Node<Key> {
    Node *left;           // 左子節點
    Node *right;          // 右子節點
    Key key;              // key 在此應用中即為賬戶地址

    // 時間戳,可用整數作為 TimeStamp
    // 例如之前可能有 using TimeStamp = int;
    TimeStamp timestamp;
};
\end{lstlisting}

時間戳擁有「只增不減」的特性,
在 LRU 應用中的最小堆積的節點的時間戳只會變大,
因此節點只會下移(shiftdown),
不會發生上移(shiftup),
也因此並不需要記錄父節點的指標。

一般的堆積在加入新元素的時候,
也有可能發生上移,
但因為時間戳「只增不減」,
新元素的時間戳一定是最大的,加入堆積尾部後,
無需上移,整個堆積依然符合最小堆積的規則。

下移的虛擬碼如下:

\begin{lstlisting}
void shiftdown(Node *node) {
    // 若沒有子節點,結束
    if (node->left == NULL && node->right == NULL) {
        return;
    }

    // 若有子節點,找出時間戳較小的
    Node *small_child = node->left;
    if (node->right != NULL &&
        node->right->timestamp < small_child->timestamp) {
        small_child = node->right;
    }

    // 若子節點時間戳小於父節點,交換父子節點
    if (small_child->timestamp < node->timestamp) {
        swap(node->key, small_child->key);
        swap(node->timestamp, small_child->timestamp);
        // 向下遞迴
        shiftdown(small_child);
    }
}
\end{lstlisting}

雜湊表則可表示成

\begin{lstlisting}
struct Info<Value> {
    TimeStamp timestamp;
    Value value;           // Value 為賬戶狀態
}
map<Key, Info> table;
\end{lstlisting}

\subsection{快取已滿}
\subsubsection{維護條件}
我們考慮快取已滿時的狀態,假設當前的堆積符合最小堆的規則,
亦即,除了根節點以外的所有節點都滿足

條件一、 父節點的時間戳都小於自己的時間戳

除此之外,我們還要維護另一個條件

條件二、 堆積的根的時間戳與雜湊表中的時間戳相同

解釋一下條件二:雜湊表中我們可以由地址去查詢到時間戳;
在堆積中的每個節點,地址也都對應到一個時間戳。
然而在整個 LRU 進行操作的過程中,除了堆積的根的時間戳,
雜湊表中的時間戳跟堆積中的時間戳並不總是同步的,
這並不要緊,只要保證堆積的根的時間戳與雜湊表中的時間戳同步,
我們就能夠順利完成 LRU 的操作。

\subsubsection{LRU get}

若快取沒有命中,不做任何事;若快取命中,更新雜湊表中的時間序。

寫為虛擬碼如下:
\begin{lstlisting}
if (table.has(address)) {   // 如果雜湊表中含有所求地址
    Info info = table.get(address);
    info.timestamp = current_timestamp++;
    table.set(address, info);
    maintain_root();
}
\end{lstlisting}

在以上過程中,我們只更新了雜湊表中的時間戳,
因此雜湊表中的時間戳在 get 之後就會與堆積中的時間戳不同,
當 get 的地址非根時無所謂,
但若恰巧 get 到了堆積的根對應的地址,
我們就得執行一個額外操作來維護前一節所提到的條件二。

這個額外操作的虛擬碼如下:

\begin{lstlisting}
void maintain_root() {
    while (heap.root.timestamp != table[heap.root.key].timestamp) {
        heap.root.timestamp = table[heap.root.key].timestamp;
        shiftdown(heap.root);
    }
}
\end{lstlisting}

maintain\_root 會更新堆積的根的時間戳,
並將根下移,若下移後堆積的新根的時間戳依然與雜湊表中的時間戳不同,重複此過程。

\subsection{LRU put}
若快取沒有命中,我們得丟棄最舊賬戶,並加入新賬戶。
具體操作中可將時間戳最小的節點,
亦即堆積的根(由於條件二,堆積的根總是最舊的)改為欲置入的新賬戶,
並執行 maintain\_root 來維護堆積規則。

若快取命中,更新雜湊表中的時間序跟賬戶狀態,並注意是否更動到堆積的根。

虛擬碼如下

\begin{lstlisting}
void lru_put(Key address, Value value) {
    if (table.has(address)) {
        // 快取命中
        Info info = table.get(address);
        info.timestamp = current_timestamp++;
        info.value = value;
        table.set(address, info);
        maintain_root();
    } else {
        // 快取失效,丟棄最舊賬戶並塞入新賬戶資訊
        table.remove(heap.root.key);
        table.set(address, { current_timestamp, value });
        heap.root.key = address;
        heap.root.timestamp = current_timestamp++;
        shiftdown(heap.root);
        maintain_root();
    }
}
\end{lstlisting}

\subsection{如何填滿快取}
無論快取是否已經填滿, get 操作的行為都相同,
但 put 時若快取失效,當快取未滿時,必須往堆積中加入新節點。
因為時間戳只增不減,將新節點塞入堆積的尾部後,
新堆積仍會符合最小堆積的規則。

至於要如何找到堆積的尾部,
我們可以維護一個 count 變數來記錄當前的堆積節點數量,
再透過讀取 count 的二進位表示式來從堆積的根走到尾部。

虛擬碼如下:

\begin{lstlisting}
Node *get_tail(int count) {
    // high_bit 取得一個數字的二進位表示式中最左側的 1 的位置
    // 例如 high_bit(0b1011) = 4, high_bit(0b0010) = 2
    int h = high_bit(count) - 1; // h 為 count 代表的節點的深度
    Node *ret = heap.root;
    while (h > 0) {
        if (count & (1 << h)) {
            ret = ret->left;
        } else {
            ret = ret->right;
        }
        h--;
    }
    return ret;
}
\end{lstlisting}

\subsection{持久化}

我們已知雜湊可輕易持久化。
而僅支援下移(shiftdown)而無需支援上移(shiftup)的堆積亦可以路徑複製來完成持久化。

堆積的 shiftdown 由根遞迴向下執行,
至多執行至葉子,時間複雜度 $O(\log n)$ ,
採用路徑複製來持久化時,
由於修改到的節點恰巧組成一條從根到葉子的路徑,
直接複製該路徑即可,耗用空間複雜度 $O(\log n)$ 。

TODO: 此處補圖,繪製 shiftdown 時路徑複製的指表操作。

\subsection{時空間複雜度分析}

假設快取大小為 $n$ 。我們分析幾個重要函式的時間複雜度。

由上節討論中,我們知道 shiftdown 的時空間複雜度是 $O(\log n)$。

maintain\_root 在最壞情況下需要將所有堆積中的時間戳都更新,此時會執行 $n$ 次 shiftdown ,時空間複雜度為 $O(n \log n)$ 。

但我們觀察到,
maintain\_root 執行的 shiftdown 次數不超過堆積中時間戳與雜湊表中時間戳相異的數量,
而每一次 LRU 的 get, put 頂多只會使堆積中時間戳與雜湊表中時間戳相異的數量增加 1 ,
總的來說,執行 shiftdown 的次數會少於等於執行 LRU get, put 的次數,
因此我們若將 maintain\_root 的成本攤銷到每一次的 get, put 操作,
一次 get, put 將分攤不到一次 shiftdown。

攤銷掉 maintain\_root 的開銷之後,
每一次 get, put 操作增加了 $O(\log n)$ 的時空間開銷,
再加上 get, put 需要查找/插入雜湊表所消耗的 $O(\log n)$ 的時空間開銷,
get, put 的時空間複雜度仍為 $O(\log n)$ 。